\documentclass{article}
\usepackage[margin=1in, top = .8in, left=.8in]{geometry}
\usepackage{comment}
\usepackage{amsmath, amssymb}
\usepackage{framed}
\usepackage{enumerate}
\usepackage{comment}
\usepackage{tikz,pgfplots}
\usepgfplotslibrary{fillbetween}
\pgfplotsset{compat=1.15}
\usepackage[hyphens]{url}

\begin{document}


\begin{center}
    \large \textbf{Homework 9}
\end{center}
    %\item[\textbf{Week 9}]
                \begin{enumerate}
                    \item The relevant values for the pairs $(i,j)$ in the doubly-indexed sum are depicted in the graph below. $$\sum_{i=1}^{5}\sum_{j=0}^{i-1} i^2j^3$$ 
                    
                % Apologies for this awful tikz graph
                \begin{center}
                \begin{tikzpicture}[baseline=2cm]
                % axes
                \draw[<->] (-1.5,0) -- (3,0) {};
                \draw[<->] (0,-1.5) -- (0,2.5) {};

                % x-axis labels
                \draw (0.5, .1) -- (0.5, -.1) {};
                \draw (1.0, .1) -- (1.0, -.1) {};
                \draw (1.5, .1) -- (1.5, -.1) {};
                \draw (2.0, .1) -- (2.0, -.1) {};
                \draw (2.5, .1) -- (2.5, -.1) {};
                \node at (0.5, -.5) {1};
                \node at (1.0, -.5) {2};
                \node at (1.5, -.5) {3};
                \node at (2.0, -.5) {4};
                \node at (2.5, -.5) {5};

                % y-axis labels
                \draw (-.1, 0.5) -- (.1, 0.5) {};
                \draw (-.1, 1.0) -- (.1, 1.0) {};
                \draw (-.1, 1.5) -- (.1, 1.5) {};
                \draw (-.1, 2.0) -- (.1, 2.0) {};
                \node at (-.5, 0.5) {1};
                \node at (-.5, 1) {2};
                \node at (-.5, 1.5) {3};
                \node at (-.5, 2) {4};

                \fill (0.5, 0) circle (0.066) {};
                \fill (1, 0) circle (0.066) {};
                \fill (1.5, 0) circle (0.066) {};
                \fill (2, 0) circle (0.066) {};
                \fill (2.5, 0) circle (0.066) {};
                
                \fill (1, 0.5) circle (0.066) {};
                \fill (1.5, 0.5) circle (0.066) {};
                \fill (2, 0.5) circle (0.066) {};
                \fill (2.5, 0.5) circle (0.066) {};
                
                \fill (1.5, 1) circle (0.066) {};
                \fill (2, 1) circle (0.066) {};
                \fill (2.5, 1) circle (0.066) {};   \fill (2, 1.5) circle (0.066) {};
                \fill (2.5, 1.5) circle (0.066) {};
           
                \fill (2.5, 2) circle (0.066) {};
                

                \node at (3.2, 0) {$i$};
                \node at (0, 2.7) {$j$};
                \end{tikzpicture}
                \end{center}
                
                 Which of the following doubly-indexed sums has the same value?
                 
                 \begin{enumerate}
                     \item $\displaystyle \sum_{j=1}^{5}\sum_{i=0}^{j-1} i^2j^3$
                     \item $\displaystyle \sum_{j=0}^{4}\sum_{i=j+1}^{5} i^2j^3$
                     \item $\displaystyle \sum_{j=1}^{4}\sum_{i=1}^{j} i^2j^3$
                     \item $\displaystyle \sum_{j=1}^{5}\sum_{i=j}^{4} i^2j^3$
                 \end{enumerate}
                    
                    \item Let  $$p(x,y) = \begin{cases} 
                        c(x^2+y^2) & \text{if $x\in {1, 2, 4}$ and $y\in{1, 3}$} \\
                        0 & \text{otherwise} \\
                        \end{cases}
                        $$
                        \begin{enumerate}
                            \item Find $c$ so that $p(x,y)$ defines a valid (joint) probability mass function.  That is, find the value of $c$ so that $\displaystyle \sum_x \sum_y p(x, y) = 1$. 
                            \item Find $\displaystyle E[XY] =\sum\sum xyp(x,y)$.
                        \end{enumerate}
                        %\item Approximate the following finite sum, for large values of $N$:
                        %$$\frac{1-1+ \frac{1}{2}-\frac{1}{3!}+\cdot \cdot \cdot + \frac{(-1)^{N-k}}{(N-k)!}} {k!}$$
                    \item Find three positive numbers whose sum is 100 and whose product is a maximum. (Once you create the objective function, there will be only one critical point satisfying the criteria --- you are not obligated to show that the product is maximum at that critical point.)
                    \item Let $\displaystyle \ell(\vec{x}:\mu, \sigma) = \ln\left(\prod_{k=1}^{n} \frac{1}{\sqrt{2\pi\sigma^2}}\exp\left(-\frac{(x_k-\mu)^2}{2\sigma^2}\right)\right)$.  Treating the values of $\vec{x} = [x_1, x_2, ..., x_n]$ as constants, at what value of $\mu$, $\sigma$ does $\ell(\vec{x}:\mu, \sigma)$ have a maximum?  You only need to find the single critical number (and argue that it is the only one) to receive full credit. Extra credit may be given for fully justifying that this is in fact a maximum.
                    \item Convert the point at $r=7$, $\theta = 2\pi/3$ to rectangular coordinates.
                    \item Find the linearization of the function $\displaystyle z=f(x,y) = \ln(x^2 +3y)$ centered at the point $(2, -1)$, and use it to estimate $f(1.97, -.9)$. 
         \item Evaluate $$\int\int_R xye^{x^2y}\,dA$$ 
         over the region $R = [0,1]\times [0,2]$. Be wise about which order of integration you choose.
         \item For each of the following, sketch the region of integration and change the order of integration.
         \begin{enumerate}
         
         \item $\displaystyle \int_0^4\int_{0}^{\sqrt{x}} f(x,y)\,dy\,dx$
         \item $\displaystyle \int_0^1\int_{2x}^4 f(x,y)\,dy\,dx$
        \item $\displaystyle \int_0^3\int_0^{\sqrt{9-y}}f(x,y)\,dx\,dy$
        \end{enumerate}
        \item Suppose that the function $f(x,y)$ defined below is a probability density function:
        $$ f(x,y) = \begin{cases} 
            ce^{x^2} & \text{if } (x,y) \in [0,1]\times [0,1] \text{ and } y \leq x\\
            0 & \text{otherwise} \\
            \end{cases}
        $$
        Find the value of $c$.
        \item For the probability density function given in the previous problem, find the probability that the pair $(x,y)$ satisfies $y \leq 2x$.
        \item A uniform probability density function is defined on the region $[0,2]\times[0,1]$. Find the probability that the pair $(x,y)$ satisfies $y \leq \sqrt{x}$ and $y \leq 2-x$.
       
       \item Suppose that a probability density function is given by 
       $$ f(x,y) = \begin{cases} 
            c(1-x^2-y^2) & \text{if } x^2+y^2\leq 1 \\
            0 & \text{otherwise} \\
            \end{cases}
        $$
        Find the probability that $x^2+y^2 \leq 0.5$

                \end{enumerate}
\end{document}