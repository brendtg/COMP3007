\documentclass{article}
\usepackage[margin=1in, top = .8in, left=.8in]{geometry}
\usepackage{comment}
\usepackage{amsmath, amssymb}
\usepackage{framed}
\usepackage{enumerate}
\usepackage{comment}
\usepackage{tikz,pgfplots}
\usepgfplotslibrary{fillbetween}
\pgfplotsset{compat=1.15}
\usepackage[hyphens]{url}

\begin{document}

\begin{center}
    \large \textbf{Homework 7}
\end{center}
    %\item[\textbf{Week 7}]
        \begin{itemize}
            %\item Sync (week 6):
            \item Part 1
                \begin{enumerate}
                    \item Determine the value of the partial sum $S_5$ for the series.
                        \begin{enumerate}
                            \item $\displaystyle \sum_{j=1}^\infty \frac{1}{j^2}$
                            \item $\displaystyle \sum_{j=1}^\infty \frac{j}{2^j}$
                        \end{enumerate}
                    \item Evaluate the sum $\displaystyle \sum_{j=1}^\infty a_j$ (or determine that it diverges) given the formula for its $n$th partial sum.
                        \begin{enumerate}
                            \item $\displaystyle S_n = 2+\frac{n}{3n+2}$
                            \item $\displaystyle S_n = n+\frac{7}{n}$
                            \item $\displaystyle S_n = \frac{n^2-4n+5}{n^3+7n-9}$
                        \end{enumerate}
                    \item Suppose the $n$th partial sum of the series $\displaystyle \sum_{j=1}^\infty a_j$ is $$S_n = \frac{n-1}{n+1}$$  Determine the value of $\displaystyle \sum_{j=1}^\infty a_j$ and determine the general term $a_n$ of the series. Note: the first part of the problem is exactly the same concept as the previous problem on this assignment. The second part, which is to reverse-engineer the formula for $S_n$ to discover the formula for $a_j$ requires you to be a bit inventive. 
                    \item Use the Divergence Theorem to determine if the series diverges or explain why the test is inconclusive.  
                        \begin{enumerate}
                            \item $\displaystyle \sum_{n=1}^\infty \frac{n^2+3}{n^2+11}$
                            \item $\displaystyle \sum_{n=2}^\infty \frac{7n}{\ln n}$
                            \item $\displaystyle \sum_{n=0}^\infty \frac{9}{55+2n}$
                        \end{enumerate}
                    \item Determine whether a term-size comparison can be made between $a_n$ and $b_n$.  Then determine whether a conclusion about the convergence or divergence of  $\displaystyle \sum_{n=1}^\infty a_n$ can be made based on this comparison and, if so, what the conclusion is.
                        \begin{enumerate}
                            \item $\displaystyle a_n =\frac{1}{e^n+n^2}$ and $\displaystyle b_n=\dfrac{1}{n^2}$
                            \item $\displaystyle a_n =\frac{1}{e^n-n^2}$ and $\displaystyle b_n=\dfrac{1}{e^n}$ 
                            \item  $\displaystyle a_n = \frac{3n-2}{2n^3+5}$ and $b_n=\dfrac{3n}{n^3}$
                            \item $\displaystyle a_n = \frac{n^2-n+5}{n^3-3n+6}$ and $b_n=\dfrac{1}{n}$
                        \end{enumerate}
                    \item In class, the following additional test for comparing series was suggested.                   \begin{framed}
                    \noindent \textbf{The Limit Comparison Test}\\
                    Suppose $a_n>0$ and $b_n>0$ for all $n$.  If $\displaystyle \lim_{n\rightarrow\infty}\frac{a_n}{b_n} = c$, where $c$ is finite and $c>0$, then the two series $\sum a_n$ and $\sum b_n$ either both converge or both diverge.
                    \end{framed} 
                    For any of the series from problem 5 for which a conclusion could not be  drawn by term-wise comparison, apply the limit comparison test and determine what (if any) conclusion can be drawn.              
                \end{enumerate}
            %\item Async (Week 7):
            \item Part 2
                \begin{enumerate}
                    \item Determine whether the series converges or diverges.
                        \begin{enumerate}
                            \item $\displaystyle \sum_{n=0}^\infty \frac{n}{5n+1}$
                            \item $\displaystyle \sum_{n=0}^\infty \frac{1000}{5n-1}$   
                            \item $\displaystyle \sum_{n=1}^\infty \frac{n^2}{\sqrt{n^7+2n+1}}$
                            \item $\displaystyle \sum_{n=1}^\infty \frac{\ln n}{n}$
                            \item $\displaystyle \sum_{n=1}^\infty \frac{\ln n^2}{n^3}$
                            \item $\displaystyle \sum_{n=2}^\infty \frac{5+3\sin n}{n-1}$
                            \item $\displaystyle \sum_{n=1}^\infty \frac{\cos^2 n}{4n^3+n^2}$
                            \item $\displaystyle \sum_{n=2}^\infty \frac{1}{\sqrt{n}\ln n}$
                            \item $\displaystyle \sum_{n=1}^\infty \frac{5}{n2^n}$
                            \item $\displaystyle \sum_{n=2}^\infty \frac{3\sqrt{n}}{\ln 5n}$
                            \item $\displaystyle \sum_{n=1}^\infty \frac{(-2)^{n-1}}{ne^n}$
                        \end{enumerate}
                    \item Find the sum.
                        \begin{enumerate}
                            \item $\displaystyle \sum_{n=1}^\infty 2\cdot \left(\frac{1}{3}\right)^{n-1}$
                            \item $\displaystyle \sum_{n=1}^\infty \frac{e}{\pi^{n-1}}$
                            \item $\displaystyle \sum_{n=4}^\infty \frac{9}{10^n}$
                            \item $\displaystyle \sum_{n=3}^\infty \frac{5^{2n-1}}{e^{3n+3}}$
                        \end{enumerate}
                    \item Find $k$ so that $\displaystyle p(n) =\frac{k\cdot2^{3n-4}}{3^{2n+3}}$ defines a probability mass function. \\  That is, find $k$ so that $\displaystyle \sum_{n=0}^\infty p(n) = 1$.
                    \item Determine the sum of the following series, including which value of $c$ it converges for.  $$\displaystyle \sum_{n=0}^\infty \frac{1}{c^n}$$
                    \item Create a power series representation of the indicated function by manipulating a known power series.
                        \begin{enumerate}
                            \item $\displaystyle f(x)=\frac{1}{1-7x}$
                            \item $\displaystyle f(x)=\frac{1}{3+x}$
                            \item $\displaystyle f(x)=\ln \sqrt{1+x^2}$
                            \item $\displaystyle f(x)=\frac{x^2}{(1-x)^3}$
                        \end{enumerate}
                \end{enumerate}
        \end{itemize}

\end{document}