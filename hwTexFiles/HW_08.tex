\documentclass{article}
\usepackage[margin=1in, top = .8in, left=.8in]{geometry}
\usepackage{comment}
\usepackage{amsmath, amssymb}
\usepackage{framed}
\usepackage{enumerate}
\usepackage{comment}
\usepackage{tikz,pgfplots}
\usepgfplotslibrary{fillbetween}
\pgfplotsset{compat=1.15}
\usepackage[hyphens]{url}

\begin{document}


\begin{center}
    \large \textbf{Homework 8}
\end{center}
    %\item[\textbf{Week 8}]
        \begin{itemize}
            %\item Sync (Week 7):
            \item Part 1
                \begin{enumerate}
                    \item Evaluate the sum.
                        \begin{enumerate}
                            \item $\displaystyle \sum_{n=2}^\infty \frac{n(-3)^n}{4^{n+1}}$
                            \item $\displaystyle \sum_{n=1}^\infty {n^2e^n}$
                            \item $\displaystyle \sum_{n=3}^\infty \frac{n(n-1)}{2^{n-2}}$
                            \item $\displaystyle \sum_{n=0}^\infty n^2\lambda^n e^{-\lambda}$ 
                        \end{enumerate}
                    \item Let $f(x) = \sin x$.  Here are the derivatives of sine and cosine. $$\frac{d}{dx}[\sin x] = \cos x \quad \text{ and } \quad \frac{d}{dx}[\cos x] = -\sin x$$
                        \begin{enumerate}
                            \item Find the 6th degree Taylor polynomial centered at $a=0$ for $\sin x$.
                            \item Estimate the value of $\sin(\pi/12)$ using this Taylor polynomial.
                            \item Write the the general term $a_n$ of the Taylor series for $\sin x$.
                        \end{enumerate}
                    \item Find the Taylor series (centered at $a=0$) for $f(x) = \cos x$.  Write the series in expanded form as well as with summation notation.  (HINT: Using the general formula for the Taylor polynomial is not the only way to solve this problem.)
                    \item Given that $\displaystyle f(x) = \sum_{k=1}^{\infty} kx^{k-1} = \frac{1}{(1-x)^2}$ and that a function defined by a power series can be differentiated by differentiating the series term by term, give a formula for $\displaystyle \sum_{k=2}^\infty k(k-1)x^{k-2}$ that does not use an infinite sum.
                \end{enumerate}
            %\item Async (Week 8):
            \item Part 2
                \begin{enumerate}
                    \item Evaluate each sum.
                        \begin{enumerate}
                            \item $\displaystyle \sum_{k=1}^\infty 3\left(\frac{1}{4}\right)^k$
                            \item $\displaystyle \sum_{k=1}^\infty (-1)^{k}\frac{3^{k+1}}{k7^k}$
                            \item $\displaystyle \frac{\pi^4}{3!}+\frac{\pi^5}{4!}+\frac{\pi^6}{5!}+\cdots$
                        \end{enumerate}
                    \item The following summations are related to the Poisson distribution.  Evaluate each sum.
                        \begin{enumerate}
                            \item Given $\displaystyle \sum_{k=0}^\infty \frac{x^k}{k!}=\exp(x)$, evaluate $\displaystyle\sum_{k=0}^\infty \frac{\lambda^k}{k!}\exp(-\lambda)$  (This is the sum of the probabilities in the Poisson probability distribution.)
                            \item Evaluate $\displaystyle \sum_{k=0}^\infty \frac{k\lambda^k}{k!}\exp(-\lambda)$ (This is the mean of the Poisson distribution.)
                            \item Evaluate $\displaystyle \sum_{k=0}^\infty. \frac{k(k-1)\lambda^k}{k!}\exp(-\lambda)$
                            \item Evaluate $\displaystyle \sum_{k=0}^\infty. \frac{k^2\lambda^k}{k!}\exp(-\lambda)$
                            \item Evaluate $\displaystyle \left( \sum_{k=0}^\infty. \frac{k^2\lambda^k}{k!}\exp(-\lambda)\right) - \left(\sum_{k=0}^\infty \frac{k\lambda^k}{k!}\exp(-\lambda)\right)^2$ (This is the variance of the Poisson distribution.)
                        \end{enumerate}
                    \item Evaluate $\displaystyle \int \frac{e^x-1}{x}\, dx$ as an infinite series.
                    \item Let $$\displaystyle f(x)= C\sum_{n=1}^\infty \frac{(-1)^nx^{3n-1}}{(n-1)!}$$  Find the constant $C$ so that $f$ defines a valid probability density function on the interval $[0,2]$.
                \end{enumerate}
        \end{itemize}

\end{document}