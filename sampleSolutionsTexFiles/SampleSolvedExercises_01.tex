\documentclass{article}
\usepackage[margin=1in, top = .8in, left=.8in]{geometry}
\usepackage{comment}
\usepackage{amsmath, amssymb}
\usepackage{framed}
\usepackage{enumerate}
\usepackage{comment}
\usepackage{tikz,pgfplots}
\pgfplotsset{compat=1.15}

\usepackage{url}
\newcommand{\R}{\mathbb{R}}
\newcommand{\N}{\mathbb{N}}
\everymath{\displaystyle}

\begin{document}
\begin{center}
\textbf{
    Sample solved exercises with solutions for Homework 1}
\end{center}
    \begin{enumerate}
        \item Complete the square in the following expression. In other words, rewrite it so it is of the form $a(x-b)^2 + k$
        $$5c(\alpha x^2 + \pi p x)$$
        \item Explain the difference between the following two expressions:
        $$\sum_{i=1}^n (y_i - \mu)$$
        and 
        $$\sum_{i=1}^n y_i - \mu.$$
        \item Consider the function $\displaystyle p(x) = \frac{1}{\sqrt{2\pi \sigma^2}}\exp\left( \frac{-x_1^2-x_2^2-x_3^2}{2\sigma^2}\right)$. Define a new function $h(x) = \ln(p(x))$. Use the laws of logs to fully simplify $h(x)$,
        \item A function can be understood as a process by which we produce outputs from a given input.  The types of things we input and output can vary greatly from function to function.  We can input single numbers, lists of numbers, or perhaps something non-numeric. When you see the notation $m:\mathbb{R}^2 \to \mathbb{R}$, we are saying that $m$ is a function that inputs two numbers and returns a single number.  So $m$ might look like \[m(a,b)=ab+\sqrt{ab}\]
        Match the following functions on the left to the descriptors on the right.
        Match the property of matter from the left column with the appropriate measurement device or technique on the right.
\begin{center}
\begin{tabular}{llll}
1.  &   $f(x,y)=(xy,yx+1)$ &   a.       &   $\R^3\to \R$\\
&&&\\
2.  &   $f(x)=(x^2-3,\ln(x+1))$    &   b. & $\{\text{words}\}\to \N$  \\
&&&\\
3.  &   $f(x,y,z)=xyz$      &   c.  &   $\R\to\R^2$\\
&&&\\
4.  &   $f(x,y)=(xy,xy,xy)$      &   d. &   $\R^2\to \R^2$\\
&&&\\
5.  &   $f(\text{word})=\text{length of word}$       &   e. &   $\R^2\to \R^3$
\end{tabular}
\end{center}
    \item Say that $p(i)$ is defined below:
                        $$ p(i) = \begin{cases} 
                        c\cdot i^2 & \text{if $1 
                        \leq i \leq 10, \quad i \in \mathbb{N}$} \\
                        0 & \text{otherwise} \\
                        \end{cases}
                        $$
                        Find the value of the constant $c$ so that $\displaystyle \sum_{i=1}^{10} p(i)$ = 1. Then, using that value of $c$, find $\displaystyle \sum_{i=1}^{10} i\cdot p(i)$ and $\displaystyle \sum_{i=1}^{10} i^2 \cdot p(i)$.
    \item Write the following series in summation notation.
                           \begin{enumerate}
                            \item $\displaystyle \frac{1}{2}-\frac{4}{4}+\frac{9}{8}-\frac{16}{16} + \ldots - \frac{100}{1024}$
                            \item $\displaystyle \frac{3}{5\cdot 2!}+\frac{5}{5^2\cdot 3!}+\frac{7}{5^3\cdot 4!} + \ldots$
            \end{enumerate}
    \end{enumerate}
    \newpage
    \begin{center}
        \textbf{\Large{Solutions}}
    \end{center}
    \begin{enumerate}
        \item Our goal is to re-write the given expression in the form $a(x-b)^2 + k$
        \begin{eqnarray*}
        5c(\alpha x^2 + \pi p x) &=& 5c\alpha (x^2+\frac{\pi p}{\alpha}x) \\ [1em]
        &=& 5c\alpha \left(x^2+\frac{\pi p}{\alpha}x+\left(\frac{\pi p}{2\alpha}\right)^2-\left(\frac{\pi p}{2\alpha}\right)^2\right) \\[1em]
        &=& 5c\alpha \left(x+\frac{\pi p}{2\alpha}\right)^2- 5c\alpha \frac{\pi^2 p^2}{4\alpha^2} 
        \end{eqnarray*}
        From this, we find that $a=5c\alpha$ and $k=- 5c\alpha \frac{\pi^2 p^2}{4\alpha^2} $
        \item The parentheses make a big difference. In the first expression, $\sum_{i=1}^n (y_i - \mu)$, the ``$-\mu$'' is part of each summand, while in the second expression, $\sum_{i=1}^n y_i - \mu$, the $\mu$ is subtracted once from the total sum. The first expression can be simplified as follows:
        \begin{eqnarray*}
        \sum_{i=1}^n (y_i - \mu) &=&
        (y_1-\mu) + (y_2 - \mu) + \ldots + (y_n - \mu)\\ 
        &=&y_1 + y_2 + \ldots + y_n - \mu - \mu - \ldots - \mu \\
        &=& \sum_{i=1}^n y_i - n\mu
        \end{eqnarray*}
        \item Using the product-to-sum law and the fact that $\ln$ and $\exp$ are inverse functions of each other, we can re-write the given function as follows:
        \begin{eqnarray*}
        \ln(p(x)) &=& \ln\left(\frac{1}{\sqrt{2\pi \sigma^2}}\exp\left( \frac{-x_1^2-x_2^2-x_3^2}{2\sigma^2}\right)\right) \\[1em]
        &=& \ln\left(\frac{1}{\sqrt{2\pi \sigma^2}}\right) + \ln\left(\exp\left( \frac{-x_1^2-x_2^2-x_3^2}{2\sigma^2}\right) \right) \\[1em]
        &=&\ln\left(\frac{1}{\sqrt{2\pi \sigma^2}}\right) +\left( \frac{-x_1^2-x_2^2-x_3^2}{2\sigma^2}\right)
        \end{eqnarray*}
        \item \begin{itemize}
            \item $1.\leftrightarrow d.$
            \item $2.\leftrightarrow c.$
            \item $3.\leftrightarrow a.$
            \item $4.\leftrightarrow e.$
            \item $5.\leftrightarrow b.$
        \end{itemize}
        \item We need the following:
        $$\sum_{i=1}^{10} c\cdot i^2 = c\sum_{i=1}^{10} i^2 = 1$$
        Searching online, we find the formula $\sum_{i=1}^n i^2 = \frac{n(n+1)(2n+1)}{6}$. 
        Substituting gives
        $$c\cdot\frac{10(10+1)(2\cdot 10+1)}{6} = c \cdot 385 =1,$$
        so $c=\frac{1}{385}$. Now that we have $c$, we can re-write $\sum_{i=1}^{10} i\cdot p(i)=\sum_{i=1}^{10} \frac{1}{385}\cdot i^3$.  This requires another internet search to find $\sum_{i=1}^n i^3 = \frac{n^2(n+1)^2}{4}$. Substituting gives $$\sum_{i=1}^{10} \frac{1}{385}\cdot i^3 = \frac{1}{385}\frac{100\cdot121}{4} = \frac{3025}{325}$$
        
        Likewise we find $\sum_{i=1}^ni^4 = \frac{n(n+1)(6n^3+9n^2+n-1)}{30}$. Substituting gives
        \[\sum_{i=1}^{10} \frac{1}{385}i^4= \frac{25333}{385}\]
        
        \item
            \begin{enumerate}
                \item We can start this series with the index $n=1$. First note that the terms of the series alternate signs, starting with a positive for the index 1 term. The expression $(-1)^{n-1}$ alternates sign (use $(-1)^n$ if the index 0 term is  negative). The numerators look like the square numbers, given by $n^2$. The denominators are the powers of 2, given by $2^n$. The last term has a numerator of $10^2$ and a denominator of $2^10$, so the last term is $n=10$. Putting it all together, the series is $$\sum_{n=1}^{10} (-1)^{n-1} \frac{n^2}{2^n}$$.
                \item The numerators of the terms of this series have a constant difference of 2, so the numerators are an arithmetic sequence. If we let the first term have index $n=1$, the numerator is the linear expression $2n+1$. The denominators have a geometric component ($5^n$) and a factorial component ($(n+1)!$). The series is an infinite series. Putting it all together, the series in summation form is 
                $$\sum_{n=1}^{\infty}  \frac{2n+1}{5^n\cdot (n+1)!}$$.
            \end{enumerate}
    \end{enumerate}
\end{document}